% ----------------------------------------------------------
% Introdução
% ----------------------------------------------------------
\chapter{Introdução} 
\label{cha:introducao}
Este capítulo descreve o contexto e a motivação do presente trabalho, bem como o problema que se pretende investigar, ressaltando-se a relevância e o contexto específico da pesquisa, além de seus objetivos, das justificativas para a proposta relacionada e suas principais contribuições.

De acordo com \citeonline{maia2016reconceptualizing} Projetar e planejar uma atividade educacional é um processo que tem grande impacto no sucesso dessa atividade \cite{maia2016reconceptualizing}.


Exemplos de citação:
% \citeonline{wazlawick2017metodologia}
% \cite{wazlawick2017metodologia}
Com autor citado no texto: Segundo \citeonline{wazlawick2017metodologia}, aconteceu isso e aquilo.

Sem autor citado no texto \cite{wazlawick2017metodologia}.

\section{Contexto e Motivação}

Nesta seção apresente o contexto no qual está inserido seu trabalho, o problema e a motivação para seu desenvolvimento.

\section{Justificativa}
Nesta seção apresente o que justifica o estudo a ser elaborado, ou seja o porquê que o tema deve ser estudado.

\section{Questões de Pesquisa}
O que você pretende resolver/entender?

\section{Objetivos}
Nesta seção o objetivo geral e os objetivos específicos devem ser relacionados em itens por meio de verbos no infinitivo. Os objetivos devem ser simples e mensuráveis.

\subsection{Objetivo Geral}
Objetivo geral: indica o que se pretende alcançar de forma ampla e está relacionado à questão principal da pesquisa.

\subsection{Objetivos Específicos}
Objetivos específicos: contribuem para alcançar o objetivo geral, apontando as etapas que levam a consecução do objetivo maior. Observe a melhor sequência lógica, estabelecendo quais assuntos precedem a outros

\section{Estrutura do trabalho}
Breve descrição de cada capítulo

